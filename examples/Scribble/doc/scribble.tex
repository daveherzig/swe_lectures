\documentclass[12pt,a4paper,german]{article}
\usepackage{babel}
\usepackage{colortbl}
\usepackage{times}
\usepackage{fancybox}
\usepackage{fancyhdr}
\usepackage[latin1]{inputenc}
\usepackage{ktsi}
\usepackage{graphicx}
\usepackage{alltt}
\usepackage{pifont}
\newcommand{\OK}{\ding{52}}
%
\newenvironment{Boxedminipage}%
  {\begin{Sbox}\begin{minipage}}
  {\end{minipage}\end{Sbox}\shadowbox{\TheSbox}}
\setlength{\headheight}{-4cm}
\addtolength{\textwidth}{2cm}
\addtolength{\textheight}{65.cm}
\setlength{\oddsidemargin}{-0.5cm}
\setlength{\evensidemargin}{0pt}
\usepackage{keywords}
%\usepackage[bookmarks]{hyperref}
\begin{document}
\section*{Zeichnungsprogramm}
Es soll ein Programm zur interaktiven Erstellung von einfachen Zeichnungen entwickelt werden.
Das Programm soll nach dem Aufstarten 
ein leeres Fenster anzeigen, in welchem der Benutzer
durch Verschieben des Mauszeigers
bei gedr\"uckter Maustaste Linienz\"uge erzeugen kann, die 
in bestimmter Strichdicke und Farbe unmittelbar der Bewegung des 
Mauszeigers folgend angezeigt werden.
\subsection*{Bestimmung der Klassen}
Textanalyse (Substantive):\\[2ex]
\begin{tabular}{llcll}
          & englisch & Verwendung & MFC-Basisklasse & Qt-Basisklasse \\
\hline
 Programm & Application & \OK & CWinApp & \\
 Erstellung & Creation &    & & \\
 Zeichnung & Drawing, Document &  \OK & (CDocument) & \\
 Aufstarten & Start-up & & & \\
 Fenster & Window, View & \OK & CFrameWnd & QWindow \\
 Zeichenfl�che & DrawWindow & \OK & \\
 Verschieben & Move &  & \\
 Mauszeiger & Mouse pointer & &\\
 Maustaste & Mouse button & &\\
 Benutzer & User &         & \\
 Linienzug & Line stroke & \OK &\\
 Bewegung & Move & & \\[4ex]
\end{tabular}
\subsection*{Klassendiagramm}
\includegraphics[width=\linewidth]{xfig/zchclass-diag}
\newpage
\subsection*{Sequenzdiagramm}
\includegraphics[width=\linewidth]{xfig/zchseq-diag}
%\subsection*{Sequenzdiagramm}
\large
\begin{alltt}
// ScribbleApp.h
#if !defined( SCRIBBLEAPP_H )
#define SCRIBBLEAPP_H

#include <afxwin.h>

\CLASS ScribbleApp: \PUBLIC CWinApp \{
\PUBLIC:
  ScribbleApp();
  \VIRTUAL BOOL InitInstance();
\};
#endif
\newpage
// DrawWindow.h
#if !defined(DRAWWINDOW_H)
#define DRAWWINDOW_H

#include <afxwin.h>
#include "Document.h"

\CLASS DrawWindow : \PUBLIC CFrameWnd \{
\PUBLIC:
   // {\em Constructor}
   DrawWindow();
   // {\em Destructor}
   \VIRTUAL ~DrawWindow();

   // constructs a frame window directly
   \INT create( LPCTSTR lpszWindowName );
   \VIRTUAL \VOID OnPaint();

\PROTECTED:
   \VOID OnLButtonDown( UINT nFlags, CPoint p );
   \VOID OnLButtonUp( UINT nFlags, CPoint p );
   \VOID OnMouseMove( UINT nFlags, CPoint p );

   DECLARE_MESSAGE_MAP()

\PRIVATE:
   Document m_document;
\};
#endif
\newpage
// Document.h: interface for the Document class.
#if !defined(DOCUMENT_H)
#define DOCUMENT_H

#include "Stroke.h"
#include <vector>

\CLASS Document\{
\PUBLIC:
  Document();
  \VIRTUAL ~Document();
  // {\em adds a new stroke}
  \VOID newStroke();
  // {\em adds a point to the last stroke}
  \VOID add( \CONST CPoint &p );
  /// {\em draws all strokes}
  \VOID draw( CDC *dc ) \CONST;
  /// {\em draws the connecting line between the last two points}
  \VOID drawLast( CDC *dc ) \CONST;

\PRIVATE:
  \TYPEDEF std::vector< Stroke > StrokeVector;
  StrokeVector m_strokes;
\};
#endif
\newpage 
// {\em Stroke.h: interface for the Stroke class.}
#if !defined(STROKE_H)
#define STROKE_H

#include <afxwin.h>
#include <vector>

\CLASS Stroke \{
\PUBLIC:
   Stroke(\UNSIGNED \SHORT nPenWidth=2, \INT color=0 );
   // {\em Copy constructor}
   Stroke( \CONST Stroke& s);
   \VIRTUAL ~Stroke();
   // {\em adds a new point}
   \VOID add( \CONST CPoint &p );
   // {\em draws the complete stroke}
   \VOID draw( CDC *pdc ) \CONST;
   // {\em draws the connection between the last two points}
   \VOID drawLast( CDC *pdc ) \CONST;

\PRIVATE:
   \TYPEDEF std::vector< CPoint > PointVector;
   PointVector m_points;
   \INT m_penWidth;
   \INT m_color;
\};
#endif 
\end{alltt}
\newpage
\begin{alltt}
// ScribbleApp.cpp: 
// \mbox  {\em Defines the class behavors for the application }

#include "ScribbleApp.h"
#include "DrawWindow.h"

// Global object
ScribbleApp theApp;

// Constructor
ScribbleApp::ScribbleApp()\{
   // empty
\}

// Initialialization
BOOL ScribbleApp::InitInstance()\{
   DrawWindow *mw = new DrawWindow;
   m_pMainWnd = mw;

//{\em initializes the window's class name and window name }
//{\em and registers default values for its style, parent}
//{\em and associated menu.}

   mw -> Create( "Scribble" ); 

   mw -> ShowWindow(m_nCmdShow);
   mw ->UpdateWindow();
   \RETURN TRUE;
\}
\newpage
// {\em DrawWindow.cpp: implementation of the DrawWindow class.}
#include "DrawWindow.h"
#include "Stroke.h"

BEGIN_MESSAGE_MAP( DrawWindow, CFrameWnd )
   ON_WM_CREATE()
   ON_WM_LBUTTONDOWN()
   ON_WM_LBUTTONUP()
   ON_WM_MOUSEMOVE()
   ON_WM_PAINT()
END_MESSAGE_MAP()

DrawWindow::DrawWindow()\{ // {\em empty} \}

DrawWindow::~DrawWindow()\{  // {\em empty} \}

\INT DrawWindow::create( LPCTSTR lpszWindowName )\{
   \RETURN CFrameWnd::Create(0, lpszWindowName );
\}

\VOID DrawWindow::OnPaint( )\{
   CClientDC dc( \THIS );
   m_document.draw(&dc);
\}

\VOID DrawWindow::OnLButtonDown( UINT, CPoint p )\{
   m_document.newStroke( );
   m_document.add(p);
   SetCapture();  // Capture the mouse until button up
\}

\VOID DrawWindow::OnLButtonUp( UINT, CPoint p )\{
   ReleaseCapture(); // Release the mouse capture
\}

\VOID DrawWindow::OnMouseMove( UINT, CPoint p )\{
   \IF( GetCapture() != \THIS )\{
      \RETURN;
   \}
   m_document.add(p);
   CClientDC dc(this);
   m_document.DrawLast(&dc);
\}
\newpage
// {\em Document.cpp: implementation of the Document class.}

#include "Document.h"

Document::Document()\{
\}

Document::~Document()\{
\}

\VOID Document::newStroke()\{
   m_strokes.push_back( Stroke() );
\}

\VOID Document::add(const CPoint &p)\{
   m_strokes.back().add( p );
\}

\VOID Document::draw( CDC *dc)\{
   StrokeVector::iterator s;
   \FOR( s=m_strokes.begin(); s!=m_strokes.end(); ++s )
      (*s).draw(dc);
\}

VOID Document::drawLast( CDC *dc)\{
   m_strokes.back().drawLast(dc);
\}
\newpage
// {\em Stroke.cpp: implementation of the Stroke class.}

#include "Stroke.h"

Stroke::Stroke(\UNSIGNED \SHORT nPenWidth, \INT color):
  m_penWidth(nPenWidth)
, m_color(color))\{ \}

Stroke::~Stroke()\{ // {\em empty} \}


// add new point
\VOID Stroke::add( \CONST CPoint &p )\{
  m_points.push_back( p );
\}
\newpage
// draw all points
\VOID Stroke::Draw( CDC *pdc ) CONST\{
   if( m_points.empty() )
      return;  // {\em There is nothing to do}

   CPen pen;
   pen.CreatePen( PS_SOLID, m_pen_width, m_color ); 
   CPen *oldPen = pdc -> SelectObject( &pen );

   PointVector::const_iterator i= m_points.begin();
   pdc -> MoveTo( *i );
   ++i;
   \WHILE( i != m_points.end() )\{
      pdc -> LineTo( *i );
      ++i;
   \}
   pdc->SelectObject( oldPen );
\}

// draw the last two points
\VOID Stroke::DrawLast( CDC *pdc ) \CONST\{
   \IF( m_points.size() < 2)
      return;  // {\em There is nothing to do}

   CPen pen;
   pen.CreatePen( PS_SOLID, m_pen_width, m_color ); 
   CPen *oldPen = pdc -> SelectObject( &pen );

   int i = m_points.size()-2;
   pdc -> MoveTo( m_points[i] );
   pdc -> LineTo( m_points[i+1] );
   pdc->SelectObject( oldPen );
\}
\end{alltt}
\end{document}
