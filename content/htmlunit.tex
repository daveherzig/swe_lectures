\subsubsection{HtmlUnit}
HtmlUnit wird als ``Browser für Java-Programme'' bezeichnet.
HtmlUnit basiert auf Junit. Das heisst, man geht grundsätzlich wie gewohnt vor:
Man erstellt eine Testklasse und fügt
 dieser Klasse
 sukzessive die entsprechenden Testmethoden hinzu. 

\newslide
Am einfachsten verwendet man 
HtmlUnit mit Maven und ergänzt das POM-File mit folgendem Element:
\begin{lstlisting}[language=xml,
   morekeywords={dependency,groupId,artifactId,version,scope}]
<dependency>
  <groupId>net.sourceforge.htmlunit</groupId>
  <artifactId>htmlunit</artifactId>
  <version>2.4</version>
</dependency>
\end{lstlisting}
Dazu muss die Junit-Version auf den Wert 4.4 gesetzt werden.
\newslide
Als Testfall 
wird man in der Regel eine Server-Verbindung kreieren und überprüfen, ob die
erwarteten Ergebnisse zurückgegeben werden:
\begin{lstlisting}[language=java]
public class SimpleWebTest {
  @Test
  public void homePage() throws Exception {
    final WebClient webClient = new WebClient();
    final HtmlPage page = (HtmlPage)webClient.
                 getPage("http://localhost:8080/hitchhikers-guide");
    assertEquals("Welcome", page.getTitleText());
  }
\end{lstlisting}
\newslide
Das folgende Beispiel zeigt, dass auch Web-Formulare getestet werden können:
\begin{lstlisting}[language=java]
  @Test
  public void submitForm() throws Exception {
    final WebClient webClient = new WebClient();
    final HtmlPage page1 = (HtmlPage)webClient.
                 getPage("http://localhost:8080/hello/login");
    final HtmlForm page1.getFormByName("loginform");
    final HtmlTextInput textfield = form.getInputByName("username");
    final HtmlSubmitInput button = form.getInputByName("submit");

    textfield.setValueAttribute("douglas");
    final HtmlPage page2=(HtmlPage)button.click();
    assertEquals( "Main", page.getTitleText() );
  }
\end{lstlisting}
