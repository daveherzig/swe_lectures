\subsubsection{JWebUnit}
JWebUnit ist ein auf Junit und HtmlUnit basierendes
Java-Test-Framework für Web-Applikationen. JWebUnit verwendet man am
einfachsten mit Maven:
\begin{lstlisting}[language=xml,
  morekeywords={dependency,groupId,artifactId,version,scope}]
<dependency>
  <groupId>net.sourceforge.jwebunit</groupId>
  <artifactId>jwebunit-htmlunit-plugin</artifactId>
  <version>2.2</version>
  <scope>test</scope>
</dependency>
 \end{lstlisting}
\newslide
Im Vergleich zu HtmlUnit ist die Verwendung einfacher:
\begin{lstlisting}[language=java]
public class SimpleWebUnitWebTest extends WebTestCase {
  public void setUp(){
    getTestContext().
         setBaseUrl("http://localhost:8080/hitchhiker");
  }
  public void testHomePage() {
    beginAt("/");
    assertTitleEquals("Welcome");
  }
}
\end{lstlisting}
\newslide
Das gilt auch für Formulare:
\begin{lstlisting}[language=java]

  public void testSubmitForm() throws Exception {
    beginAt("/hello/login");
    assertFormPresent("loginform");
    assertFormElementPresent("username");
    setTextField("username", "douglas");
    submit();
    assertTitleEquals( "Main" );
  }
\end{lstlisting}
\newslide
Weitere Vereinfachungen betreffen das Testen von Tabellen und HTML-Elementen
und das Navigieren durch die Seiten.
\begin{lstlisting}[language=java]
  assertElementPresent("pageTitle");
  assertTextInElement("pageTitle", "Hello World");
\end{lstlisting}
In diesem Beispiel wird geprüft, ob ein Element
mit dem Id-Attribut ``pageTitle'' existiert und ob darin der
Text ``Hello World'' enthalten ist.

\newslide
Störend ist hier wie schon bei HtmlUnit, 
dass zuerst ein WebContainer gestartet
werden muss, bevor man die Tests durchführt.

\newslide
Dank Jetty können diese Tests aber wie normale Unit-Tests durchgeführt
werden:
\begin{lstlisting}[language=java]
import org.mortbay.jetty.Server;
public class SimpleWebUnitWebTest extends WebTestCase {
  private Server server;
	
  public void setUp() throws Exception{
    String context="/hitchhiker";
    // Port 0 means "assign arbitrarily port number"
    server = new Server(0);
    server.addHandler(new WebAppContext("src/main/webapp", context));
    server.start();
    int port = server.getConnectors()[0].getLocalPort();
    getTestContext().setBaseUrl("http://localhost:"+port+context);
  }
..
}
\end{lstlisting}
\newslide
Dazu muss lediglich die POM-Datei etwas angepasst werden:
\begin{lstlisting}[language=xml,
  morekeywords={dependency,groupId,artifactId,version,scope}]
    <dependency>
      <groupId>org.mortbay.jetty</groupId>
      <artifactId>jetty</artifactId>
      <version>6.1.14</version>
      <scope>test</scope>
    </dependency>
    <dependency>
      <groupId>org.mortbay.jetty</groupId>
      <artifactId>jsp-2.1</artifactId>
      <version>6.1.14</version>
      <scope>test</scope>
    </dependency> 
\end{lstlisting}