\chapter{API Development}

An application programming interface (API) is an interface that
defines interactions between multiple software applications or
mixed hardware-software intermediaries. It defines the kinds of calls
or requests that can be made, how to make them, the data formats
that should be used, the conventions to follow, etc. It can also
provide extension mechanisms so that users can extend existing
functionality in various ways and to varying degrees. An API can be
entirely custom, specific to a component, or designed based on an
industry-standard to ensure interoperability. Through information hiding,
APIs enable modular programming, allowing users to use the interface
independently of the implementation.

\section{FAIR Data}
FAIR data are data which meet principles of findability,
accessibility, interoperability, and reusability.
The acronym and principles were defined in a March 2016 paper in the
journal Scientific Data by a consortium of scientists and organizations.\\

The FAIR principles emphasize machine-actionability (i.e., the capacity of
computational systems to find, access, interoperate, and reuse data
with none or minimal human intervention) because humans increasingly
rely on computational support to deal with data as a result of the
increase in volume, complexity, and creation speed of data.\\

\begin{itemize}
\item \textbf{Findable}\\
The first step in (re)using data is to find them. Metadata and data
should be easy to find for both humans and computers. Machine-readable
metadata are essential for automatic discovery of datasets and services,
so this is an essential component of the FAIRification process.
\item \textbf{Accessible}\\
Once the user finds the required data, they need to know how they can be
accessed, possibly including authentication and authorisation.
\item \textbf{Interopable}\\
The data usually need to be integrated with other data. In addition, the data
need to interoperate with applications or workflows for analysis, storage,
and processing.
\item \textbf{Reusable}\\
The ultimate goal of FAIR is to optimise the reuse of data. To achieve
this, metadata and data should be well-described so that they can be
replicated and/or combined in different settings.
\end{itemize}



\section{REST}

\section{GraphQL}
