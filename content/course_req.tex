% -----------------------------------------------------------------------

\section{Required Software Environment}
The main topic of this course is software engineering and how
to build high quality software systems. The complete course does
not depend on any specific platforms nor technologies.\\
To fulfill the exercises it is (of course) needed to use
some specific technologies. These technologies could be used
on any operating system (e.g. Windows, OSX, Linux). In the
following section there is an overview of these technologies and
a short description.\\

\begin{itemize}
\item Java Development Kit 15\\
(https://www.oracle.com/ch-de/java/technologies/javase-downloads.html)
\item IDE Integrated Development Environment\\
There are 3 very popular IDEs:
\begin{itemize}
\item Eclipse (https://www.eclipse.org/)
\item IntelliJ (https://www.jetbrains.com/de-de/idea/)
\item Netbeans (https://netbeans.org/)
\end{itemize}
All of these environments have their pros and cons. Within this course, IntelliJ
will be used
\item Texteditor Atom (https://atom.io/)
A simple text editor is one of the most powerful tools. Atom will be used, as
this editor is available on most common operating systems.
\item Apache ANT (https://ant.apache.org/)
Build Tool
\item Apache Maven (https://maven.apache.org/)
Build Tool
\item Apache Tomcat (http://tomcat.apache.org/)
Simple Servlet Engine
\item Git (https://git-scm.com/)
Source Code management
\item MySQL (https://www.mysql.com/)
A relational database
\item StarUML (https://staruml.io/)
Graphical UML editor
\end{itemize}

\vspace{1mm}

There will be several other frameworks which will be used during the course.
These will be installed on demand. The ones listed above should be installed
prior the course start.
