% -----------------------------------------------------------------------
\chapter{Object Oriented Programming}
%-----------------------------------------------------------------------
\section{Introduction}
Object-oriented programming is a programming paradigm based on the
concept of \emph{objects}, which can contain data, in the form of
fields (attributes or properties), and code, in the form of procedures
(methods).\\
The key concepts in OOP are:
\begin{itemize}
\item Abstraction (objects, classes)
\item Data Encapsulation
\item Inheritance
\item Polymorphism
\end{itemize}
The following chapter will give a brief overview of the key concepts
in object oriented programming. This is of course not a course about
object oriented programming. But it makes perfectly sense to understand
the key concepts as they are also used in several areas of software
engineering.

\section{Languages}
There are many object oriented programming languages available.
Each language has its advantages and disadvantages. The most popular
languages are:
\begin{itemize}
\item C++
\item Java
\item .Net
\item Python
\item ...and many more!
\end{itemize}

\section{Abstraction}
Abstraction means using simple things to represent complexity.
We all know how to turn the TV on, but we do not need to know how
it works in order to enjoy it. In OOP, abstraction means simple
things like objects, classes, and variables represent more complex
underlying code and data. This is important because it lets avoid
repeating the same work multiple times.

\begin{lstlisting}
public class TV {
  private Color color;
  private int size;
  public void turnOn(){ // some complex code }
  public void turnOff(){ // some complex code }
  public void changeChannel(){ // some complex code }
}

TV tv1 = new TV();
TV tv2 = new TV();
\end{lstlisting}
TV is the class definition. \verb|tv1| and \verb|tv2| are the
objects. They both have their own color and size.

\section{Data Encapsulation}
This is the practice of keeping fields within a class private,
then providing access to them via public methods. It's a protective
barrier that keeps the data and code safe within the class itself.
This way, we can re-use objects like code components or variables
without allowing open access to the data system-wide.

\begin{lstlisting}
public class Circle {
  private double radius;
  public void setRadius(double radius) {this.radius = radius;}
  public void getRadius() {return radius;}
}

Circle circle = new Circle();
circle.setRadius(10); // correct way
circle.radius = 10; // compilation error, radius is private
\end{lstlisting}

\section{Inheritance}
This is a special feature of Object Oriented Programming.
It lets programmers create new classes that share some of the
attributes of existing classes. This lets us build on previous work
without reinventing the wheel.

\begin{lstlisting}
public class Doctor {
  protected String name;
  public void doctorDetails() { // do some stuff }
}

public class Surgeon extends Doctor {
  public void surgeonDetails() { // do some stuff }
}

Surgeon jamesMiller = new Surgeon();
jamesMiller.surgeonDetails();
jamesMiller.doctorDetails();
\end{lstlisting}

The object \verb|jamesMiller| is a doctor and a surgeon.

\section{Polymorphism}
This OOP concept lets programmers use the same word to mean
different things in different contexts. One form of polymorphism
is method overloading. That's when different meanings are implied
by the code itself. The other form is method overriding. That's when
the different meanings are implied by the values of the supplied
variables.

\begin{lstlisting}
public abstract class DBConnection {
  public abstract void connect();
};

public class OracleConnection extends DBConnection {
  public void connect() { // connect to oracle db }
};

public class SQLServerConnection extends DBConnection {
  public void connect() { // connect to sql server db }
};

DBConnection connection1 = new OracleConnection();
DBConnection connection2 = new SQLServerConnection();
connection1.connect();
connection2.connect();
\end{lstlisting}

Both objects, \verb|connection1| and \verb|connection2| are of
type \verb|DBConnection|. During runtime, the \verb|connection1| object
will point to a \verb|OracleConnection| object, the
\verb|connection2| object will point to a \verb|SQLServerConnection| object.
When calling the \verb|connect| method o both objects, a different method
will be called.

\section{Exercises}
\begin{enumerate}
\item Write a class called \verb|Util|. This class should contain a method
\verb|createSampleDNA| which will create a sample DNA sequence of
a given length.\\
The program should be callable from the command line. One parameter
could be provided, which specifies the length of the sequence. If
no parameter is provided, the default length is 100.
\item Extend your \verb|Util| class with a method which counts the
number of A, G, C and T in a given sequence. The sequence will also
be provided as parameter.
\item Implement the data loader component. Create a project in your
preferred development environment and create on class \verb|Loader|.
This class should fulfill the following steps:
\begin{itemize}
\item The program should be called with a properties file as parameter.
This file will contain information like db location, username and password.
\item The program will read the data file line by line
\item The program will insert the data into the database
\item Please ensure that the program is running with an acceptable performance and resource consumption
\end{itemize}
\item If the data is loaded, please try to answer the following questions by
using SQL statements:
\begin{itemize}
\item How many gene information entries are available?
\item How many genes have the type \emph{protein-coding}?
\item How many genes have the type \emph{protein-coding} and are
related to the specie \emph{Human}?
\end{itemize}
\end{enumerate}

\subsection{SQL Help}
\begin{lstlisting}
drop database if exists XZY;

create database XYZ;

use XZY;

create table TTT (
  column1 varchar(256),
  column2 int,
  ...,
  PRIMARY KEY(column1),
  FOREIGN KEY(column2) REFERENCES TABLE_X(column_x)
);

CREATE USER 'dataloader'@'localhost' IDENTIFIED BY 'password';
GRANT ALL PRIVILEGES on XZY.* TO 'dataloader'@'localhost';
FLUSH PRIVILEGES;
\end{lstlisting}
