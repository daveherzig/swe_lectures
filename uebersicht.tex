\section*{Overview}
\subsection*{Learning goals}
\begin{itemize}
\item Systematic approach to gain knowledge in theories, methods,
and tools to design and build a software that meets the specifications
efficiently, cost-effectively, and ensures quality.
%Die wesentlichen Merkmale und Konzepte des Software-Engineerings
% kennen und anwenden können um bedienerfreundliche und wartbare
% Software-Systeme zu erstellen.
\end{itemize}
\newslide
Content:
\begin{itemize}
\item Introduction
\item Software project planning
\item Software development life cycle (SDLC):
  \begin{itemize}
  \item Sequential, iterative and agile models
  \end{itemize}
\item Software Requirements Analysis
\item Software Design:
  \begin{itemize}
  \item Design patterns
  \item Use Cases: Docker, JDBC, Hibernate, Spring
  \item Unit-Testing
  \end{itemize}
\item Configuration management
  \begin{itemize}
  \item Git
\end{itemize}
\item Build Tools
  \begin{itemize}
  \item Ant, Maven
  \end{itemize}
\item Testing
  \begin{itemize}
  \item Code analyzer, logging
  \item Unit tests
  \item Performance tests, memory tests, profiling
  \item User interface tests
  \end{itemize}
%\item Testen: Methoden und Werkzeuge, Dokumentation
%  \begin{itemize}
%  \item Code Analyzer, Logging
%  \item Unit-Tests %, Integrationstests
%  \item Speichertest, Performance-Test, Profiling
%  \item GUI-Test, Web-Test
%  \item Behaviour Driven, FIT
%  \end{itemize}
\end{itemize}
%
\newslide
\subsection*{Certificate of achivement}
The final grade will be calculated based on two inputs:
\begin{itemize}
\item \verb|30%| Course project
\item \verb|70%| Written exam
\end{itemize}

%Das Modul wird durch eine
%schriftliche Prüfung (Dauer 90 Minuten)
%abgeschlossen. Erlaubte Unterlagen: alles ausser elektronische Geräte.
%Die Prüfungsfragen orientieren sich an den im Unterricht behandelten Übungen.
%Zusätzlich ist eine Fallstudie durchzuführen, die mit einer
%Semesternote
%bewertet wird.

%Die Gewichtung der Noten ist: 25\% Semesternote, 75\% Modulprüfungsnote.

%{\bfseries Hinweis:}  Zu den Übungen werden keine Musterlösungen verteilt.
\newslide
\subsection*{Timetable}
\begin{tabularx}{\linewidth}{|l|X|}
\hline
TO BE DEFINED & TO BE DEFINED\\
%18. September & Einführung, Anforderungsspezifikation\\
%25.        &  \\
%2.  Oktober      &    \\
%9.   &   \hfill Abgabe \\
%16.  &   UML, Vorgehensmodelle \\
%23.   & Projektplanung, Docker\\
%30.   & Konfigurationsmanagement\\
%6. November        &  Versionsverwaltung \\
%13.        &  Unit-Testing \\
%20.         &  JDBC, Hibernate\\
\hline
\end{tabularx}
%
%\begin{tabularx}{\linewidth}{|l|X|}
%\hline
%16. September & Einführung, Vorgehensmodelle \\
%23.        &  UML, Anforderungsspezifikation\\
%30.        & \\
%7.  Oktober      &    \\
%14.   &   \hfill Abgabe \\
%\hline
%\ifslides
%21.   & \multicolumn{1}{Al|}{Projektarbeit} \\
%\else
%21.   & \multicolumn{1}{Al|}{}\\
%28.   & \multicolumn{1}{Al|}{}\\
%4. November        &  \multicolumn{1}{Al|}{Projektarbeit} \\
%11.        & \multicolumn{1}{Al|}{}\\
%\fi
%18.         &  \multicolumn{1}{Al|}{}\\
%\hline
%25.         &  Konfigurationsmanagement\\
%2. Dezember &  Projektplanung, Docker\\
%9.         &  JDBC\\
%16.         &  Hibernate\\
%\hline
%\ifslides
%21.   & \multicolumn{1}{Al|}{\raisebox{1.5ex}{Weihnachtsferien}} \\
%\else
%23.   & \multicolumn{1}{Al|}{}\\
%30.   & \multicolumn{1}{Al|}{\raisebox{1.5ex}[-1.5ex]{Weihnachtsferien}} \\
%\fi
%\hline
%6. Januar        &  Spring\\
%\hline
%\end{tabularx}

%%% mode: latex
%%% TeX-master: "kurs"
%%% End:
